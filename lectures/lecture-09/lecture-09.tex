\documentclass{beamer}\usepackage[]{graphicx}\usepackage[]{color}
%% maxwidth is the original width if it is less than linewidth
%% otherwise use linewidth (to make sure the graphics do not exceed the margin)
\makeatletter
\def\maxwidth{ %
  \ifdim\Gin@nat@width>\linewidth
    \linewidth
  \else
    \Gin@nat@width
  \fi
}
\makeatother

\definecolor{fgcolor}{rgb}{1, 0.894, 0.769}
\newcommand{\hlnum}[1]{\textcolor[rgb]{0.824,0.412,0.118}{#1}}%
\newcommand{\hlstr}[1]{\textcolor[rgb]{1,0.894,0.71}{#1}}%
\newcommand{\hlcom}[1]{\textcolor[rgb]{0.824,0.706,0.549}{#1}}%
\newcommand{\hlopt}[1]{\textcolor[rgb]{1,0.894,0.769}{#1}}%
\newcommand{\hlstd}[1]{\textcolor[rgb]{1,0.894,0.769}{#1}}%
\newcommand{\hlkwa}[1]{\textcolor[rgb]{0.941,0.902,0.549}{#1}}%
\newcommand{\hlkwb}[1]{\textcolor[rgb]{0.804,0.776,0.451}{#1}}%
\newcommand{\hlkwc}[1]{\textcolor[rgb]{0.78,0.941,0.545}{#1}}%
\newcommand{\hlkwd}[1]{\textcolor[rgb]{1,0.78,0.769}{#1}}%
\let\hlipl\hlkwb

\usepackage{framed}
\makeatletter
\newenvironment{kframe}{%
 \def\at@end@of@kframe{}%
 \ifinner\ifhmode%
  \def\at@end@of@kframe{\end{minipage}}%
  \begin{minipage}{\columnwidth}%
 \fi\fi%
 \def\FrameCommand##1{\hskip\@totalleftmargin \hskip-\fboxsep
 \colorbox{shadecolor}{##1}\hskip-\fboxsep
     % There is no \\@totalrightmargin, so:
     \hskip-\linewidth \hskip-\@totalleftmargin \hskip\columnwidth}%
 \MakeFramed {\advance\hsize-\width
   \@totalleftmargin\z@ \linewidth\hsize
   \@setminipage}}%
 {\par\unskip\endMakeFramed%
 \at@end@of@kframe}
\makeatother

\definecolor{shadecolor}{rgb}{.97, .97, .97}
\definecolor{messagecolor}{rgb}{0, 0, 0}
\definecolor{warningcolor}{rgb}{1, 0, 1}
\definecolor{errorcolor}{rgb}{1, 0, 0}
\newenvironment{knitrout}{}{} % an empty environment to be redefined in TeX

\usepackage{alltt}
\usepackage{../371g-slides}
\title{Dummy Variables}
\subtitle{Lecture 9}
\author{STA 371G}
\IfFileExists{upquote.sty}{\usepackage{upquote}}{}
\begin{document}
  
  
  

  \frame{\maketitle}

  % Show outline at beginning of each section
  \AtBeginSection[]{ 
    \begin{frame}<beamer>
      \tableofcontents[currentsection]
    \end{frame}
  }

  %%%%%%% Slides start here %%%%%%%

  \begin{darkframes}
  
    \begin{darkframes}    
    
    
    \begin{frame}
      \fontsize{9}{9}\selectfont
      Predicting the fuel economy (MPG) for different car models of `70s.
      
      \begin{center}
        \includegraphics[width=2.8in]{bmw} \\
      \end{center} \pause
      
      \begin{columns}[onlytextwidth]
        \column{.5\textwidth}
          \begin{itemize}
            \item Cylinders
            \item Displacement
            \item Horsepower
          \end{itemize}
        \column{.5\textwidth}
          \begin{itemize}
            \item Weight
            \item Acceleration
            \item Year (After 1975 or not)
          \end{itemize}
      \end{columns}
    \end{frame}
    
    
    
    
    
    \begin{frame}[fragile]{Exploring the data}
      \fontsize{9}{9}\selectfont
      Let's display the first 5 rows (and all columns).
\begin{knitrout}
\definecolor{shadecolor}{rgb}{0.137, 0.137, 0.137}\begin{kframe}
\begin{alltt}
\hlstd{> }\hlstd{auto_mpg[}\hlnum{1}\hlopt{:}\hlnum{5}\hlstd{,]}
\end{alltt}
\begin{verbatim}
# A tibble: 5 � 7
    MPG Cylinders Displacement    HP Weight Acceleration After1975
  <dbl>     <int>        <dbl> <int>  <int>        <dbl>     <chr>
1    18         8          307   130   3504         12.0        No
2    15         8          350   165   3693         11.5        No
3    18         8          318   150   3436         11.0        No
4    16         8          304   150   3433         12.0        No
5    17         8          302   140   3449         10.5        No
\end{verbatim}
\end{kframe}
\end{knitrout}
      \pause
      No??? What the... What to do with that? \pause
      
      Maybe just omit the ``After1975'' column?
  
    \end{frame}
    
    
    
    \begin{frame}[fragile]%{Exploring the data}
      \fontsize{9}{9}\selectfont
      \note{This slide is to get familiar with boxplots. Conclusion: Being manufactured after 1975 matters}
\begin{knitrout}
\definecolor{shadecolor}{rgb}{0.137, 0.137, 0.137}\begin{kframe}
\begin{alltt}
\hlstd{> }\hlkwd{boxplot}\hlstd{(MPG} \hlopt{~} \hlstd{After1975,} \hlkwc{data}\hlstd{=auto_mpg,} \hlkwc{ylab}\hlstd{=}\hlstr{"MPG"}\hlstd{,}
\hlstd{+ }              \hlkwc{xlab}\hlstd{=}\hlstr{"After 1975"}\hlstd{,} \hlkwc{col}\hlstd{=}\hlstr{'darkgray'}\hlstd{)}
\end{alltt}
\end{kframe}
\input{C:/temp/figures/unnamed-chunk-4-1.tikz}

\end{knitrout}
      
    \end{frame}
    
    
    
    
    \begin{frame}[fragile]{Regression with categorical variables}
      \fontsize{9}{9}\selectfont
      Can we go ahead and run a regression anyway? \pause
\begin{knitrout}
\definecolor{shadecolor}{rgb}{0.137, 0.137, 0.137}\begin{kframe}
\begin{alltt}
\hlstd{> }  \hlstd{model} \hlkwb{<-} \hlkwd{lm}\hlstd{(MPG} \hlopt{~} \hlstd{Cylinders} \hlopt{+}\hlstd{Displacement} \hlopt{+} \hlstd{HP}
\hlstd{+ }              \hlopt{+} \hlstd{Weight} \hlopt{+} \hlstd{Acceleration} \hlopt{+} \hlstd{After1975,}
\hlstd{+ }              \hlkwc{data}\hlstd{=auto_mpg)}
\hlstd{> }  \hlkwd{summary}\hlstd{(model)}\hlopt{$}\hlstd{r.squared}
\end{alltt}
\begin{verbatim}
[1] 0.776176
\end{verbatim}
\end{kframe}
\end{knitrout}
      \pause
      
      R was able to handle the ``After1975'' column, which is a \alert{categorical variable} (or a \alert{factor} as R calls them).
    \end{frame}
    
    
    
    
    \begin{frame}[fragile]{Dummy variables}
      \fontsize{9}{9}\selectfont
\begin{knitrout}
\definecolor{shadecolor}{rgb}{0.137, 0.137, 0.137}\begin{kframe}
\begin{alltt}
\hlstd{> }  \hlkwd{round}\hlstd{(}\hlkwd{summary}\hlstd{(model)}\hlopt{$}\hlstd{coefficients,} \hlnum{2}\hlstd{)}
\end{alltt}
\begin{verbatim}
             Estimate Std. Error t value Pr(>|t|)
(Intercept)     42.19       2.37   17.81     0.00
Cylinders       -0.58       0.36   -1.62     0.11
Displacement     0.01       0.01    0.94     0.35
HP              -0.02       0.01   -1.35     0.18
Weight          -0.01       0.00   -8.33     0.00
Acceleration     0.04       0.11    0.32     0.75
After1975Yes     4.36       0.40   10.85     0.00
\end{verbatim}
\end{kframe}
\end{knitrout}
    R has created a \alert{dummy variable}, ``After1975Yes.'' \pause 
    A dummy variable is always 0 or 1, indicating the absence or presence of some categorical effect.
    
    \end{frame}
    
    
    
    \begin{frame}[fragile]{Dummy variables}
      \fontsize{9}{9}\selectfont
      ``After1975Yes'' is 1 whenever ``After1975'' is a ``Yes,'' and 0 otherwise.
    
      \begin{table}[!b]
        {\carlitoTLF % Use monospaced lining figures
        \begin{tabularx}{\textwidth}{rrrrr}
           
           MPG &  ... & Acceleration & After1975 & After1975Yes\\ 
          \toprule
            ... & ... & ... & ... & ... \\
            25 & ... & 13.5 & No & 0 \\
            33 & ... & 17.5 & No & 0 \\
            28 & ... & 15.5 & Yes & 1 \\
            25 & ... & 16.9 & Yes & 1 \\
            ... & ... & ... & ... & ... \\
          \bottomrule
            
        \end{tabularx}}
        
      \end{table} 
      
      \pause
      Notice that we do not have a ``After1975No'' variable. 
      
      It would cause problems because it would be perfectly correlated with ``After1975Yes.''
      \end{frame}
      
      
      
      \begin{frame}[fragile]{Regression with categorical variables}
        \fontsize{9}{9}\selectfont
        Our model contains some statistically insignificant variables. 
        
        Your task is to start omiting them one by one. What is the $R^2$ in your final model?
        \lc
      
      \end{frame}
      
      
      
      \begin{frame}[fragile]{Regression with categorical variables}
        \fontsize{9}{9}\selectfont
\begin{knitrout}
\definecolor{shadecolor}{rgb}{0.137, 0.137, 0.137}\begin{kframe}
\begin{alltt}
\hlstd{> }\hlstd{model} \hlkwb{<-} \hlkwd{lm}\hlstd{(MPG} \hlopt{~}  \hlstd{HP} \hlopt{+} \hlstd{Weight} \hlopt{+} \hlstd{After1975,}
\hlstd{+ }                   \hlkwc{data}\hlstd{=auto_mpg)}
\hlstd{> }\hlkwd{summary}\hlstd{(model)}\hlopt{$}\hlstd{r.squared}
\end{alltt}
\begin{verbatim}
[1] 0.7745063
\end{verbatim}
\begin{alltt}
\hlstd{> }\hlkwd{round}\hlstd{(}\hlkwd{summary}\hlstd{(model)}\hlopt{$}\hlstd{coefficients,} \hlnum{2}\hlstd{)}
\end{alltt}
\begin{verbatim}
             Estimate Std. Error t value Pr(>|t|)
(Intercept)     41.71       0.78   53.15     0.00
HP              -0.02       0.01   -2.30     0.02
Weight          -0.01       0.00  -13.84     0.00
After1975Yes     4.33       0.40   10.83     0.00
\end{verbatim}
\end{kframe}
\end{knitrout}
        \pause
        Horsepower seems to be already capturing the information in Cylinders, Displacement and Acceleration. 
      
      \end{frame}
      
      
      \begin{frame}[fragile]{Interpretation of the $\beta$ of the dummy variable}
        \fontsize{9}{9}\selectfont
        Consider this:
        \begin{itemize}
          \item Model A and B have the same HP and Weight.
          \item Model A was manufactured before 1975, whereas B was manufactured after 1975.
          \item Our model's prediction for Model A is 21 MPG.
          \item What is the prediction for Model B?
        \end{itemize} 
          \lc
      \end{frame}
      
      
      \begin{frame}[fragile]{Interpretation of the $\beta$ of the dummy variable}
        \fontsize{9}{9}\selectfont
          Our ``reference level'' is the cars manufactured before 1975. \bigskip
          
          For the same Weight and HP, our MPG prediction for a car manufactured after 1975 is always exactly 4.33 higher compared to its reference.  \bigskip
          
          There are other coding schemes too, where the reference is chosen differently.
          
          
          
          
      \end{frame}
      
      
      \begin{frame}[fragile]{Why not to create two separate models?}
        \fontsize{9}{9}\selectfont
        
        *** THIS DATA DO NOT SUPPORT THIS ARGUMENT BECAUSE OF THE NONLINEARITY ***
        
         First regress a model only for the cars manufactured before 1975.
\begin{knitrout}
\definecolor{shadecolor}{rgb}{0.137, 0.137, 0.137}\begin{kframe}
\begin{alltt}
\hlstd{> }        \hlstd{Weight_B} \hlkwb{<-} \hlstd{auto_mpg}\hlopt{$}\hlstd{Weight[auto_mpg}\hlopt{$}\hlstd{After1975}\hlopt{==}\hlstr{'No'}\hlstd{]}
\hlstd{> }        \hlstd{HP_B}     \hlkwb{<-} \hlstd{auto_mpg}\hlopt{$}\hlstd{HP[auto_mpg}\hlopt{$}\hlstd{After1975}\hlopt{==}\hlstr{'No'}\hlstd{]}
\hlstd{> }        \hlstd{MPG_B}    \hlkwb{<-} \hlstd{auto_mpg}\hlopt{$}\hlstd{MPG[auto_mpg}\hlopt{$}\hlstd{After1975}\hlopt{==}\hlstr{'No'}\hlstd{]}
\hlstd{> }        
\hlstd{> }        \hlstd{model_B} \hlkwb{<-} \hlkwd{lm}\hlstd{(MPG_B} \hlopt{~}  \hlstd{HP_B} \hlopt{+} \hlstd{Weight_B)}
\end{alltt}
\end{kframe}
\end{knitrout}
          
          
      \end{frame}
      
      \begin{frame}[fragile]{Why not to create two separate models?}
        \fontsize{9}{9}\selectfont
\begin{knitrout}
\definecolor{shadecolor}{rgb}{0.137, 0.137, 0.137}\begin{kframe}
\begin{alltt}
\hlstd{> } \hlkwd{summary}\hlstd{(model_B)}
\end{alltt}
\begin{verbatim}

Call:
lm(formula = MPG_B ~ HP_B + Weight_B)

Residuals:
    Min      1Q  Median      3Q     Max 
-7.2199 -1.7683 -0.0254  1.7336  6.9173 

Coefficients:
              Estimate Std. Error t value Pr(>|t|)    
(Intercept) 37.1585134  0.6664457   55.76   <2e-16 ***
HP_B        -0.0145377  0.0083065   -1.75   0.0818 .  
Weight_B    -0.0050048  0.0003896  -12.85   <2e-16 ***
---
Signif. codes:  0 '***' 0.001 '**' 0.01 '*' 0.05 '.' 0.1 ' ' 1

Residual standard error: 2.49 on 177 degrees of freedom
Multiple R-squared:   0.82,	Adjusted R-squared:  0.818 
F-statistic: 403.2 on 2 and 177 DF,  p-value: < 2.2e-16
\end{verbatim}
\end{kframe}
\end{knitrout}
          
          
      \end{frame}
      
      
      
      
      
      \begin{frame}[fragile]{Why not to create two separate models?}
        \fontsize{9}{9}\selectfont
        
        
        
        Now regress a model only for the cars manufactured after 1975.
\begin{knitrout}
\definecolor{shadecolor}{rgb}{0.137, 0.137, 0.137}\begin{kframe}
\begin{alltt}
\hlstd{> }        \hlstd{Weight_A} \hlkwb{<-} \hlstd{auto_mpg}\hlopt{$}\hlstd{Weight[auto_mpg}\hlopt{$}\hlstd{After1975}\hlopt{==}\hlstr{'Yes'}\hlstd{]}
\hlstd{> }        \hlstd{HP_A}     \hlkwb{<-} \hlstd{auto_mpg}\hlopt{$}\hlstd{HP[auto_mpg}\hlopt{$}\hlstd{After1975}\hlopt{==}\hlstr{'Yes'}\hlstd{]}
\hlstd{> }        \hlstd{MPG_A}    \hlkwb{<-} \hlstd{auto_mpg}\hlopt{$}\hlstd{MPG[auto_mpg}\hlopt{$}\hlstd{After1975}\hlopt{==}\hlstr{'Yes'}\hlstd{]}
\hlstd{> }        
\hlstd{> }        \hlstd{model_A} \hlkwb{<-} \hlkwd{lm}\hlstd{(MPG_A} \hlopt{~}  \hlstd{HP_A} \hlopt{+} \hlstd{Weight_A)}
\end{alltt}
\end{kframe}
\end{knitrout}
          
          
      \end{frame}
      
      \begin{frame}[fragile]{Why not to create two separate models?}
        \fontsize{9}{9}\selectfont
\begin{knitrout}
\definecolor{shadecolor}{rgb}{0.137, 0.137, 0.137}\begin{kframe}
\begin{alltt}
\hlstd{> } \hlkwd{summary}\hlstd{(model_A)}
\end{alltt}
\begin{verbatim}

Call:
lm(formula = MPG_A ~ HP_A + Weight_A)

Residuals:
   Min     1Q Median     3Q    Max 
-8.869 -2.724 -0.379  2.265 13.229 

Coefficients:
              Estimate Std. Error t value Pr(>|t|)    
(Intercept) 52.0736406  1.1142331  46.735  < 2e-16 ***
HP_A        -0.0716222  0.0184989  -3.872 0.000145 ***
Weight_A    -0.0066577  0.0007383  -9.018  < 2e-16 ***
---
Signif. codes:  0 '***' 0.001 '**' 0.01 '*' 0.05 '.' 0.1 ' ' 1

Residual standard error: 3.982 on 209 degrees of freedom
Multiple R-squared:  0.7296,	Adjusted R-squared:  0.727 
F-statistic:   282 on 2 and 209 DF,  p-value: < 2.2e-16
\end{verbatim}
\end{kframe}
\end{knitrout}
          
          
      \end{frame}
      
      
      
      
      \begin{frame}[fragile]{What if there are more than two categories?}
        \fontsize{8}{8}\selectfont
\begin{knitrout}
\definecolor{shadecolor}{rgb}{0.137, 0.137, 0.137}\begin{kframe}
\begin{alltt}
\hlstd{> }\hlstd{auto_mpg_all[}\hlnum{1}\hlopt{:}\hlnum{5}\hlstd{,]}
\end{alltt}
\begin{verbatim}
# A tibble: 5 � 8
    MPG Cylinders Displacement    HP Weight Acceleration After1975 Origin
  <dbl>     <int>        <dbl> <int>  <int>        <dbl>     <chr>  <chr>
1    18         8          307   130   3504         12.0        No     US
2    15         8          350   165   3693         11.5        No     US
3    18         8          318   150   3436         11.0        No     US
4    16         8          304   150   3433         12.0        No     US
5    17         8          302   140   3449         10.5        No     US
\end{verbatim}
\begin{alltt}
\hlstd{> }\hlkwd{levels}\hlstd{(}\hlkwd{as.factor}\hlstd{(auto_mpg_all}\hlopt{$}\hlstd{Origin))}
\end{alltt}
\begin{verbatim}
[1] "EU" "JP" "US"
\end{verbatim}
\end{kframe}
\end{knitrout}
        \pause
        Let's first see if ``Origin'' makes a difference.
      
      \end{frame}
      
      \begin{frame}[fragile]%{Exploring the data}
        \fontsize{9}{9}\selectfont
\begin{knitrout}
\definecolor{shadecolor}{rgb}{0.137, 0.137, 0.137}\begin{kframe}
\begin{alltt}
\hlstd{> }\hlkwd{boxplot}\hlstd{(MPG} \hlopt{~} \hlstd{Origin,} \hlkwc{data}\hlstd{=auto_mpg_all,} \hlkwc{ylab}\hlstd{=}\hlstr{"MPG"}\hlstd{,}
\hlstd{+ }              \hlkwc{xlab}\hlstd{=}\hlstr{"Origin"}\hlstd{,} \hlkwc{col}\hlstd{=}\hlstr{'darkgray'}\hlstd{)}
\end{alltt}
\end{kframe}
\input{C:/temp/figures/unnamed-chunk-13-1.tikz}

\end{knitrout}
      
    \end{frame}
      
      
      
      \begin{frame}[fragile]{Regression with categorical variables}
        \fontsize{9}{9}\selectfont
\begin{knitrout}
\definecolor{shadecolor}{rgb}{0.137, 0.137, 0.137}\begin{kframe}
\begin{alltt}
\hlstd{> }\hlstd{omodel} \hlkwb{<-} \hlkwd{lm}\hlstd{(MPG} \hlopt{~}  \hlstd{HP} \hlopt{+} \hlstd{Weight} \hlopt{+} \hlstd{After1975} \hlopt{+} \hlstd{Origin,}
\hlstd{+ }                   \hlkwc{data}\hlstd{=auto_mpg_all)}
\hlstd{> }\hlkwd{round}\hlstd{(}\hlkwd{summary}\hlstd{(omodel)}\hlopt{$}\hlstd{coefficients,}\hlnum{3}\hlstd{)}
\end{alltt}
\begin{verbatim}
             Estimate Std. Error t value Pr(>|t|)
(Intercept)    40.182      0.874  45.961    0.000
HP             -0.028      0.010  -2.837    0.005
Weight         -0.005      0.000 -10.815    0.000
After1975Yes    4.334      0.393  11.033    0.000
OriginJP        1.001      0.612   1.635    0.103
OriginUS       -1.593      0.562  -2.834    0.005
\end{verbatim}
\end{kframe}
\end{knitrout}
        \pause
        For the origin variable, R has chosen ``EU'' as the base, created a dummy variable for JP and US each.
      
      \end{frame}
      
      
      
      \begin{frame}[fragile]{Regression with categorical variables}
        \fontsize{9}{9}\selectfont
        While dealing with categorical variables, we look at the significance of the categorical variable as a whole.
        \bigskip  
        
        Unless all the dummy variables are insignificant, we do not omit the column of that categorical variable.
      
      \end{frame}
      
      
      
      \begin{frame}[fragile]{Assumptions}
        \fontsize{9}{9}\selectfont
        What are the issues with this model?
        \lc
\begin{knitrout}
\definecolor{shadecolor}{rgb}{0.137, 0.137, 0.137}\begin{kframe}
\begin{alltt}
\hlstd{> }\hlkwd{plot}\hlstd{(}\hlkwd{predict.lm}\hlstd{(omodel),} \hlkwd{resid}\hlstd{(omodel),} \hlkwc{col}\hlstd{=}\hlstr{'green'}\hlstd{,} \hlkwc{main}\hlstd{=}\hlstr{''}\hlstd{)}
\end{alltt}
\end{kframe}
\input{C:/temp/figures/unnamed-chunk-15-1.tikz}

\end{knitrout}
      
      \end{frame}
      
      
      
      \begin{frame}[fragile]{Assumptions}
        \fontsize{9}{9}\selectfont
        What about normality?
        
\begin{knitrout}
\definecolor{shadecolor}{rgb}{0.137, 0.137, 0.137}\begin{kframe}
\begin{alltt}
\hlstd{> }  \hlkwd{hist}\hlstd{(}\hlkwd{resid}\hlstd{(omodel),} \hlkwc{col}\hlstd{=}\hlstr{'green'}\hlstd{,} \hlkwc{main}\hlstd{=}\hlstr{''}\hlstd{)}
\end{alltt}
\end{kframe}
\input{C:/temp/figures/unnamed-chunk-16-1.tikz}

\end{knitrout}
      
      \end{frame}
      
      
      \begin{frame}[fragile]{Assumptions}
        \fontsize{9}{9}\selectfont
        What about normality?
        
\begin{knitrout}
\definecolor{shadecolor}{rgb}{0.137, 0.137, 0.137}\begin{kframe}
\begin{alltt}
\hlstd{> }  \hlkwd{qqnorm}\hlstd{(}\hlkwd{resid}\hlstd{(omodel),} \hlkwc{col}\hlstd{=}\hlstr{'green'}\hlstd{,} \hlkwc{main}\hlstd{=}\hlstr{''}\hlstd{)}
\end{alltt}
\end{kframe}
\input{C:/temp/figures/unnamed-chunk-17-1.tikz}

\end{knitrout}
      
      \end{frame}
      
 
  \end{darkframes}

\end{document}
