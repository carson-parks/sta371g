\documentclass{beamer}\usepackage[]{graphicx}\usepackage[]{color}
%% maxwidth is the original width if it is less than linewidth
%% otherwise use linewidth (to make sure the graphics do not exceed the margin)
\makeatletter
\def\maxwidth{ %
  \ifdim\Gin@nat@width>\linewidth
    \linewidth
  \else
    \Gin@nat@width
  \fi
}
\makeatother

\definecolor{fgcolor}{rgb}{1, 0.894, 0.769}
\newcommand{\hlnum}[1]{\textcolor[rgb]{0.824,0.412,0.118}{#1}}%
\newcommand{\hlstr}[1]{\textcolor[rgb]{1,0.894,0.71}{#1}}%
\newcommand{\hlcom}[1]{\textcolor[rgb]{0.824,0.706,0.549}{#1}}%
\newcommand{\hlopt}[1]{\textcolor[rgb]{1,0.894,0.769}{#1}}%
\newcommand{\hlstd}[1]{\textcolor[rgb]{1,0.894,0.769}{#1}}%
\newcommand{\hlkwa}[1]{\textcolor[rgb]{0.941,0.902,0.549}{#1}}%
\newcommand{\hlkwb}[1]{\textcolor[rgb]{0.804,0.776,0.451}{#1}}%
\newcommand{\hlkwc}[1]{\textcolor[rgb]{0.78,0.941,0.545}{#1}}%
\newcommand{\hlkwd}[1]{\textcolor[rgb]{1,0.78,0.769}{#1}}%
\let\hlipl\hlkwb

\usepackage{framed}
\makeatletter
\newenvironment{kframe}{%
 \def\at@end@of@kframe{}%
 \ifinner\ifhmode%
  \def\at@end@of@kframe{\end{minipage}}%
  \begin{minipage}{\columnwidth}%
 \fi\fi%
 \def\FrameCommand##1{\hskip\@totalleftmargin \hskip-\fboxsep
 \colorbox{shadecolor}{##1}\hskip-\fboxsep
     % There is no \\@totalrightmargin, so:
     \hskip-\linewidth \hskip-\@totalleftmargin \hskip\columnwidth}%
 \MakeFramed {\advance\hsize-\width
   \@totalleftmargin\z@ \linewidth\hsize
   \@setminipage}}%
 {\par\unskip\endMakeFramed%
 \at@end@of@kframe}
\makeatother

\definecolor{shadecolor}{rgb}{.97, .97, .97}
\definecolor{messagecolor}{rgb}{0, 0, 0}
\definecolor{warningcolor}{rgb}{1, 0, 1}
\definecolor{errorcolor}{rgb}{1, 0, 0}
\newenvironment{knitrout}{}{} % an empty environment to be redefined in TeX

\usepackage{alltt}
\usepackage{../371g-slides}
\title{Time Series: Autocorrelation}
\subtitle{Lecture 18}
\author{STA 371G}
\IfFileExists{upquote.sty}{\usepackage{upquote}}{}
\begin{document}
  
  

  \frame{\maketitle}

  % Show outline at beginning of each section
  \AtBeginSection[]{ 
    \begin{frame}<beamer>
      \tableofcontents[currentsection]
    \end{frame}
  }

  %%%%%%% Slides start here %%%%%%%

  \begin{darkframes}
    
    \begin{frame}{Predicting oil prices}
      \fontsize{8}{8}\selectfont
      \begin{center}
        \includegraphics[width=2.8in]{pumpjack} \\
      \end{center} 
      
      
      \begin{columns}[onlytextwidth]
        \column{.3\textwidth}
            \begin{center}
              \begin{tabular}{ll}
              \hline
                 & Oil price (\$) \\
              \hline
              1/1/2013 & 112.98  \\
              1/1/2014 & 107.94 \\
              1/1/2015 & 55.38 \\
              1/1/2016 & 36.85 \\
              1/1/2017 & 55.05 \\
              1/1/2018 & ? \\
              \hline 
            \end{tabular}
          \end{center} \pause
              
        \column{.7\textwidth}
          \begin{itemize}
            \item You think it is more likely to be around \$50-\$70 than \$100-\$120? \pause
            \item Otherwise would be too much of an increase?  \pause
            \item Then next year's price depends on this year's?  \pause
          \end{itemize}
      \end{columns}
    \end{frame}
    
    
    
    
    \begin{frame}{Time series}
      \fontsize{9}{9}\selectfont
        In a \alert{time series,} data are not necessarily independent. (Often it is not!) \pause
        \bigskip
        
        Time series:
        \begin{itemize}
          \item A sequence of measurements of the same variable collected over time.
          \item The measurements are made at regular time intervals
          
          \item Most common time series: weekly, monthly, quarterly and yearly.
          \item The variances are not necessarily constant over time either.
        \end{itemize}
    
    \end{frame}
    
    
    
    \begin{frame}{Time series - some examples}
      \fontsize{9}{9}\selectfont
      \begin{itemize}
        \item S\&P 500 index (or any other stock price)
        \item iPhone sales worldwide
        \item U.S. unemployment rate
        \item U.S. inflation rate
        \item Crime rate in Austin
      \end{itemize} \pause
      \bigskip
      Any of these could be measured at weekly, monthly, yearly etc. intervals; and each would be a separate time series.
    \end{frame}
  
  
    \begin{frame}[fragile]{Oil Prices 1979-2004}
      \fontsize{8}{8}\selectfont
\begin{knitrout}
\definecolor{shadecolor}{rgb}{0.137, 0.137, 0.137}\begin{kframe}
\begin{alltt}
\hlcom{# Convert the data into a time series object    }
\hlstd{price} \hlkwb{<-} \hlkwd{ts}\hlstd{(oil}\hlopt{$}\hlstd{price,} \hlkwc{start}\hlstd{=}\hlnum{1979}\hlstd{,} \hlkwc{frequency} \hlstd{=} \hlnum{1}\hlstd{)}
\hlcom{# Frequency: # of data points per year}
\hlkwd{plot}\hlstd{(price)}
\end{alltt}
\end{kframe}
\input{/tmp/figures/unnamed-chunk-2-1.tikz}

\end{knitrout}
    \end{frame}
  
    
    
    \begin{frame}[fragile]{Oil Prices 1979-2004}
     \fontsize{9}{9}\selectfont
      We argued that oil prices are not independent year-over-year. \pause
      
      In order to predict the oil price in a given year, can we use the previous year's price? \pause
      \bigskip
      
      \begin{center}
      $y_t$: The oil price at the end of the year $t$ \pause
      \end{center}
      \bigskip
      
      \begin{center}
          \begin{tabular}{lll}
          \hline
            $t$ & $y_t$ &  $y_{t-1}$\\
          \hline
          \ldots & \ldots & \ldots \\
          1999	& 16.56 & 11.91 \\
          2000 &	27.39 & 16.56  \\
          2001	& 23 & 27.39 \\
          2002	& 22.81 & 23 \\
          \ldots & \ldots & \ldots \\
          \hline 
        \end{tabular}
      \end{center}
      \pause
      
      $y_{t-1}$ column is obtained by shifting $y_t$ by 1. \pause
      
      The \alert{lag} between $y_t$ and $y_{t-1}$ is one time-step.
    
    \end{frame}
    
    
    \begin{frame}[fragile]{Compute one-lag time series}
     \fontsize{8}{8}\selectfont
\begin{knitrout}
\definecolor{shadecolor}{rgb}{0.137, 0.137, 0.137}\begin{kframe}
\begin{alltt}
\hlcom{# Create lag 1 time series.}
\hlstd{priceL1} \hlkwb{<-} \hlkwd{lag}\hlstd{(price,} \hlkwc{k}\hlstd{=}\hlopt{-}\hlnum{1}\hlstd{)}
\hlcom{# Put them together}
\hlstd{price_all} \hlkwb{<-} \hlkwd{cbind}\hlstd{(}\hlkwc{price}\hlstd{=price,} \hlkwc{priceL1}\hlstd{=priceL1)}
\hlstd{price_all[}\hlnum{1}\hlopt{:}\hlnum{5}\hlstd{,]}
\end{alltt}
\begin{verbatim}
     price priceL1
[1,] 25.10      NA
[2,] 37.42   25.10
[3,] 35.75   37.42
[4,] 31.83   35.75
[5,] 29.08   31.83
\end{verbatim}
\end{kframe}
\end{knitrout}
      \pause
      \bigskip
      
        priceL1 in the first row is NA because we did not have data from 1978 to put under $y_{t-1}$ column of 1979.
    \end{frame}
    
    
    
    
    \begin{frame}{Linear regression model}
      The simple linear regression model is:
      
      \begin{center} 
        \[
          y_t = \beta_0 + \beta_1 y_{t-1} + \epsilon_t
        \] 
      \end{center} \pause
      \bigskip
      
      Note that we obtained our predictor from the response itself!
    
    \end{frame}
    
    
    \begin{frame}[fragile]%{Linear regression model}
      \fontsize{8}{8}\selectfont
      When we use such a model, we expect to see a linear relation between the predictor and the response. Let's see if there is such a relation! \pause
\begin{knitrout}
\definecolor{shadecolor}{rgb}{0.137, 0.137, 0.137}\begin{kframe}
\begin{alltt}
\hlkwd{plot}\hlstd{(price} \hlopt{~} \hlstd{priceL1,} \hlkwc{xy.labels}\hlstd{=F,} \hlkwc{xy.lines}\hlstd{=F,} \hlkwc{col}\hlstd{=}\hlstr{'green'}\hlstd{)}
\end{alltt}
\end{kframe}
\input{/tmp/figures/unnamed-chunk-4-1.tikz}

\end{knitrout}
      \pause
      Indeed, the oil prices seem to be correlated with its first lag! \pause This is called \alert{autocorrelation.}
      
    \end{frame}
    
    
    \begin{frame}[fragile]%{Linear regression model}
      \fontsize{8}{8}\selectfont
\begin{knitrout}
\definecolor{shadecolor}{rgb}{0.137, 0.137, 0.137}\begin{kframe}
\begin{alltt}
  \hlstd{model} \hlkwb{<-} \hlkwd{lm}\hlstd{(price} \hlopt{~} \hlstd{priceL1,} \hlkwc{data}\hlstd{=price_all)}
  \hlkwd{summary}\hlstd{(model)}
\end{alltt}
\begin{verbatim}

Call:
lm(formula = price ~ priceL1, data = price_all)

Residuals:
     Min       1Q   Median       3Q      Max 
-11.9046  -2.9505  -0.8162   1.6303  12.4595 

Coefficients:
            Estimate Std. Error t value Pr(>|t|)    
(Intercept)   5.8724     3.8389   1.530 0.139722    
priceL1       0.7605     0.1642   4.632 0.000116 ***
---
Signif. codes:  0 '***' 0.001 '**' 0.01 '*' 0.05 '.' 0.1 ' ' 1

Residual standard error: 5.454 on 23 degrees of freedom
  (2 observations deleted due to missingness)
Multiple R-squared:  0.4827,	Adjusted R-squared:  0.4602 
F-statistic: 21.46 on 1 and 23 DF,  p-value: 0.0001164
\end{verbatim}
\end{kframe}
\end{knitrout}
      \pause
      The predictor is statistically significant! 
      
      What we have is an first-order autoregressive, \alert{AR(1)}, model.
    \end{frame}
      
      
    \begin{frame}[fragile]{AR(2) model}
      \fontsize{8}{8}\selectfont
      Let's try to add one more lag.
\begin{knitrout}
\definecolor{shadecolor}{rgb}{0.137, 0.137, 0.137}\begin{kframe}
\begin{alltt}
\hlcom{# Create lag 2 time series.}
\hlstd{priceL2} \hlkwb{<-} \hlkwd{lag}\hlstd{(price,} \hlkwc{k}\hlstd{=}\hlopt{-}\hlnum{2}\hlstd{)}
\hlcom{# Put them together}
\hlstd{price_all} \hlkwb{<-} \hlkwd{cbind}\hlstd{(}\hlkwc{price}\hlstd{=price,} \hlkwc{priceL1}\hlstd{=priceL1,} \hlkwc{priceL2}\hlstd{=priceL2)}
\hlstd{price_all[}\hlnum{1}\hlopt{:}\hlnum{5}\hlstd{,]}
\end{alltt}
\begin{verbatim}
     price priceL1 priceL2
[1,] 25.10      NA      NA
[2,] 37.42   25.10      NA
[3,] 35.75   37.42   25.10
[4,] 31.83   35.75   37.42
[5,] 29.08   31.83   35.75
\end{verbatim}
\end{kframe}
\end{knitrout}
      \pause
      The model then becomes:
      $$
        y_t = \beta_0 + \beta_1 y_{t-1} + \beta_2 y_{t-2} + \epsilon_t
      $$
    \end{frame}
	
	
	
    \begin{frame}[fragile]{AR(2) model}
      \fontsize{8}{8}\selectfont
\begin{knitrout}
\definecolor{shadecolor}{rgb}{0.137, 0.137, 0.137}\begin{kframe}
\begin{alltt}
  \hlstd{model} \hlkwb{<-} \hlkwd{lm}\hlstd{(price} \hlopt{~} \hlstd{priceL1} \hlopt{+} \hlstd{priceL2,} \hlkwc{data}\hlstd{=price_all)}
  \hlkwd{summary}\hlstd{(model)}
\end{alltt}
\begin{verbatim}

Call:
lm(formula = price ~ priceL1 + priceL2, data = price_all)

Residuals:
     Min       1Q   Median       3Q      Max 
-10.6861  -3.0937   0.7269   2.3375  10.9071 

Coefficients:
            Estimate Std. Error t value Pr(>|t|)    
(Intercept)   7.1749     3.7363   1.920 0.068505 .  
priceL1       0.8427     0.2073   4.064 0.000557 ***
priceL2      -0.1646     0.2094  -0.786 0.440530    
---
Signif. codes:  0 '***' 0.001 '**' 0.01 '*' 0.05 '.' 0.1 ' ' 1

Residual standard error: 4.911 on 21 degrees of freedom
  (4 observations deleted due to missingness)
Multiple R-squared:  0.5411,	Adjusted R-squared:  0.4974 
F-statistic: 12.38 on 2 and 21 DF,  p-value: 0.0002807
\end{verbatim}
\end{kframe}
\end{knitrout}
    \end{frame}
    
    
    
    \begin{frame}[fragile]{Autocorrelation Function}  
      \fontsize{8}{8}\selectfont
      priceL2 is not statistically significant. \pause
      
      Can we determine the number of lags to include in the model without trying one by one?
      \pause
      
      The \alert{Autocorrelation Function (ACF)} plots the correlation between the series and each of its lags.
\begin{knitrout}
\definecolor{shadecolor}{rgb}{0.137, 0.137, 0.137}\begin{kframe}
\begin{alltt}
\hlkwd{acf}\hlstd{(price)}
\end{alltt}
\end{kframe}
\input{/tmp/figures/unnamed-chunk-8-1.tikz}

\end{knitrout}
    \end{frame}
    
    
    
    \begin{frame}[fragile]{Stationarity assumption} 
      \fontsize{8}{8}\selectfont
      AR models (and many time series models) assume the stationarity of the series. \pause
      \bigskip
      
      A time series is stationary if
      \begin{itemize}
        \item the mean, $E[y_t]$, is the same over time
        \item the variance of $y_t$ is the same over time
        \item the correlation between $y_{t}$ and $y_{t-h}$ is the same over time.
      \end{itemize}
    
    \end{frame}
    
    
    
    \begin{frame}[fragile]{Stationarity assumption}
    \fontsize{9}{9}\selectfont
    To check on the stationarity of a time series, we use the Augmented Dickey-Fuller test. \pause
    \bigskip
\begin{knitrout}
\definecolor{shadecolor}{rgb}{0.137, 0.137, 0.137}\begin{kframe}
\begin{alltt}
\hlkwd{library}\hlstd{(}\hlstr{'tseries'}\hlstd{)}
\hlkwd{adf.test}\hlstd{(price)}
\end{alltt}
\begin{verbatim}

	Augmented Dickey-Fuller Test

data:  price
Dickey-Fuller = -0.28178, Lag order = 2, p-value = 0.9844
alternative hypothesis: stationary
\end{verbatim}
\end{kframe}
\end{knitrout}
      \pause
      The null hypothesis is that the series is ``explosive'' (non-stationary). \pause
      
      Since the p-value is very high, we cannot reject the null hypothesis.
    \end{frame}
      
    
    
    \begin{frame}[fragile]{Stationarity assumption}
    \fontsize{9}{9}\selectfont
      Trend and seasonality in a time series violate stationarity. \pause
      \bigskip
      
      However, many time series have either trend or seasonality, and often both! \pause
      \bigskip
      
      Let's look at Microsoft's revenue over years...
      
      
    \end{frame}
    
    
    
    \begin{frame}[fragile]{Increasing trend of Microsoft}
    \fontsize{8}{8}\selectfont
\begin{knitrout}
\definecolor{shadecolor}{rgb}{0.137, 0.137, 0.137}\begin{kframe}
\begin{alltt}
\hlcom{# Convert the data into a time series object    }
\hlstd{revenue} \hlkwb{<-} \hlkwd{ts}\hlstd{(microsoft}\hlopt{$}\hlstd{revenue,} \hlkwc{start}\hlstd{=}\hlnum{1995}\hlstd{,} \hlkwc{frequency} \hlstd{=} \hlnum{1}\hlstd{)}
\hlcom{# Frequency: # of data points per year}
\hlkwd{plot}\hlstd{(revenue)}
\end{alltt}
\end{kframe}
\input{/tmp/figures/unnamed-chunk-10-1.tikz}

\end{knitrout}
    \end{frame}
    
    
    \begin{frame}[fragile]{Increasing trend of Microsoft}
    \fontsize{9}{9}\selectfont
      Let's also verify the non-stationarity of the data through the ADF test. \pause
    \bigskip
\begin{knitrout}
\definecolor{shadecolor}{rgb}{0.137, 0.137, 0.137}\begin{kframe}
\begin{alltt}
\hlkwd{adf.test}\hlstd{(revenue)}
\end{alltt}
\begin{verbatim}

	Augmented Dickey-Fuller Test

data:  revenue
Dickey-Fuller = -0.44992, Lag order = 2, p-value = 0.9771
alternative hypothesis: stationary
\end{verbatim}
\end{kframe}
\end{knitrout}
      \pause
      
      Again, since the p-value is very high, we cannot reject the null hypothesis.
    \end{frame}
    
    
    \begin{frame}[fragile]{Increasing trend of Microsoft}
    \fontsize{9}{9}\selectfont
      Microsoft's revenue is certainly increasing. But the amount of increase each year seems to be relatively constant.
\begin{knitrout}
\definecolor{shadecolor}{rgb}{0.137, 0.137, 0.137}\begin{kframe}
\begin{alltt}
\hlcom{# Create lag 1 time series.}
\hlstd{revenueL1} \hlkwb{<-} \hlkwd{lag}\hlstd{(revenue,} \hlkwc{k}\hlstd{=}\hlopt{-}\hlnum{1}\hlstd{)}
\hlcom{# Look at the increase (first difference) each year}
\hlstd{revenue_increase} \hlkwb{<-} \hlstd{revenue} \hlopt{-} \hlstd{revenueL1}
\hlcom{# Put them together}
\hlstd{revenue_all} \hlkwb{<-} \hlkwd{cbind}\hlstd{(}\hlkwc{revenue}\hlstd{=revenue,} \hlkwc{revenueL1}\hlstd{=revenueL1,}
                     \hlkwc{revenue_increase}\hlstd{=revenue_increase)}
\hlstd{revenue_all[}\hlnum{1}\hlopt{:}\hlnum{8}\hlstd{,]}
\end{alltt}
\begin{verbatim}
     revenue revenueL1 revenue_increase
[1,]    6.10        NA               NA
[2,]    8.67      6.10             2.57
[3,]   11.36      8.67             2.69
[4,]   14.48     11.36             3.12
[5,]   19.75     14.48             5.27
[6,]   22.96     19.75             3.21
[7,]   25.30     22.96             2.34
[8,]   28.37     25.30             3.07
\end{verbatim}
\end{kframe}
\end{knitrout}
    \end{frame}
    
    \begin{frame}[fragile]{Increasing trend of Microsoft}
      \fontsize{9}{9}\selectfont
\begin{knitrout}
\definecolor{shadecolor}{rgb}{0.137, 0.137, 0.137}\begin{kframe}
\begin{alltt}
\hlkwd{plot}\hlstd{(revenue_increase,} \hlkwc{col}\hlstd{=}\hlstr{'green'}\hlstd{,} \hlkwc{ylab}\hlstd{=}\hlstr{'Increase in revenue'}\hlstd{)}
\end{alltt}
\end{kframe}
\input{/tmp/figures/unnamed-chunk-13-1.tikz}

\end{knitrout}
    
    \pause
    Still, slightly increaseing trend, but better.   
    \end{frame}
    
    
    
    
    \begin{frame}[fragile]{Increasing trend of Microsoft}
    \fontsize{9}{9}\selectfont
      Let's see what ADF test has to say. \pause
    \bigskip
\begin{knitrout}
\definecolor{shadecolor}{rgb}{0.137, 0.137, 0.137}\begin{kframe}
\begin{alltt}
\hlkwd{adf.test}\hlstd{(revenue_increase)}
\end{alltt}
\begin{verbatim}

	Augmented Dickey-Fuller Test

data:  revenue_increase
Dickey-Fuller = -3.0968, Lag order = 2, p-value = 0.1545
alternative hypothesis: stationary
\end{verbatim}
\end{kframe}
\end{knitrout}
      \pause
      
      Still cannot reject the null hypothesis and further transformation is required. But let's move on to model the yearly increase in revenue.
    \end{frame}
    
      
    
    
    \begin{frame}[fragile]{Autocorrelation Function of the increase in revenue}  
      \fontsize{8}{8}\selectfont
      
\begin{knitrout}
\definecolor{shadecolor}{rgb}{0.137, 0.137, 0.137}\begin{kframe}
\begin{alltt}
\hlkwd{acf}\hlstd{(revenue_increase)}
\end{alltt}
\end{kframe}
\input{/tmp/figures/unnamed-chunk-15-1.tikz}

\end{knitrout}
    \end{frame}
    
    
    \begin{frame}[fragile]{Autocorrelation Function of the increase in revenue}       \fontsize{9}{9}\selectfont
      What should we expect?
      \begin{itemize}
        \item There does not seem to be a very strong autocorrelation in the revenue increase time series.
        \item The autocorrelation with the second lag is higher than the first one.
      \end{itemize}
      
    \end{frame}
    
    
     \begin{frame}[fragile]
     \fontsize{8}{8}\selectfont
\begin{knitrout}
\definecolor{shadecolor}{rgb}{0.137, 0.137, 0.137}\begin{kframe}
\begin{alltt}
\hlstd{revenue_increaseL1} \hlkwb{<-} \hlkwd{lag}\hlstd{(revenue_increase,} \hlkwc{k}\hlstd{=}\hlopt{-}\hlnum{1}\hlstd{)}
\hlstd{revenue_increaseL2} \hlkwb{<-} \hlkwd{lag}\hlstd{(revenue_increase,} \hlkwc{k}\hlstd{=}\hlopt{-}\hlnum{2}\hlstd{)}
\hlstd{rev_inc_all} \hlkwb{<-} \hlkwd{cbind}\hlstd{(}\hlkwc{revenue_increase} \hlstd{= revenue_increase,}
                     \hlkwc{revenue_increaseL1} \hlstd{= revenue_increaseL1,}
                     \hlkwc{revenue_increaseL2} \hlstd{= revenue_increaseL2)}
\hlstd{rev_inc_all[}\hlnum{1}\hlopt{:}\hlnum{5}\hlstd{,]}
\end{alltt}
\begin{verbatim}
     revenue_increase revenue_increaseL1 revenue_increaseL2
[1,]             2.57                 NA                 NA
[2,]             2.69               2.57                 NA
[3,]             3.12               2.69               2.57
[4,]             5.27               3.12               2.69
[5,]             3.21               5.27               3.12
\end{verbatim}
\end{kframe}
\end{knitrout}
    
     \end{frame}
     
     
     \begin{frame}[fragile]
     \fontsize{8}{8}\selectfont
\begin{knitrout}
\definecolor{shadecolor}{rgb}{0.137, 0.137, 0.137}\begin{kframe}
\begin{alltt}
\hlstd{model_rev_inc} \hlkwb{<-} \hlkwd{lm}\hlstd{(revenue_increase} \hlopt{~} \hlstd{revenue_increaseL1}
                      \hlopt{+} \hlstd{revenue_increaseL2,} \hlkwc{data}\hlstd{=rev_inc_all)}
\hlkwd{summary}\hlstd{(model_rev_inc)}
\end{alltt}
\begin{verbatim}

Call:
lm(formula = revenue_increase ~ revenue_increaseL1 + revenue_increaseL2, 
    data = rev_inc_all)

Residuals:
    Min      1Q  Median      3Q     Max 
-4.9423 -1.6555 -0.1413  1.0997  5.1529 

Coefficients:
                   Estimate Std. Error t value Pr(>|t|)   
(Intercept)         6.59190    1.70754   3.860  0.00154 **
revenue_increaseL1 -0.08454    0.24354  -0.347  0.73331   
revenue_increaseL2 -0.41570    0.26843  -1.549  0.14231   
---
Signif. codes:  0 '***' 0.001 '**' 0.01 '*' 0.05 '.' 0.1 ' ' 1

Residual standard error: 2.613 on 15 degrees of freedom
  (4 observations deleted due to missingness)
Multiple R-squared:  0.1391,	Adjusted R-squared:  0.02435 
F-statistic: 1.212 on 2 and 15 DF,  p-value: 0.3251
\end{verbatim}
\end{kframe}
\end{knitrout}
     \end{frame}
    
    
    \begin{frame}[fragile]
     \fontsize{9}{9}\selectfont
        This model could be used* to predict the increase in the revenue, instead of the revenue itself.
        \bigskip
        
        The predicted increase could be added on top of the revenue at $t-1$ to predict the revenue in $t$.

     \end{frame}



    
  \end{darkframes}
\end{document}
